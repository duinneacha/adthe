\chapter{Introduction}
\label{chap:intro}
\lhead{\emph{Introduction}}
This chapter should comprise around 1000 words and introduces your project. Here you are setting the scene, remember the reader may know nothing about your project at this stage (other than the abstract). N.B. The sections outlined in this document are suggested, some projects will have a greater or lesser emphasis on different sections or may change titles and some will have to add other sections to provide context or detail.
% Putting in comments within the TeX file can be really useful in making notes for yourself and dumping text that you intend to edit later

\section{Motivation}
Why is it important to do a project on this topic? This should cover your key motivation for this. For example an excellent student from 2016 noticed a large number of homeless sleeping rough in Cork and was motivated to develop a system that load balanced the homeless shelters to try to accommodate the maximum number of homeless. This section can include the personal pronoun but the rest of the report should be third person passive, this is the case with most technical reports! For example here it is fine to say "... I decided to develop and app to help ...".

\section{Contribution}
This document contributes to the Projects Cost Control of Nimbus in CIT, however the principles behind this can easily be implemented in any Research Centre.
 
There are several modules of the Web Development Computer Science degree in CIT and during my internship at the Nimbus Research Centre that helped me with the planning of this project.

From the outset, the project has involved a lot of planning. The author sat down with Nimbus and developed a list of requirements that the project entailed and developed use cases and user stories from this – these were key items within the Requirements Engineering SOFT7007 module from the first semester of Year Two.

There were learned elements of the Object Orientated Analysis and Design (SOFT7005) course that have formed part of this project. This approach was used to design the Project Budget Data Import Module and the Web Scraping modules.

The Web Scraping module is written in Python and controlled via chron, these are elements of the Systems Scripting (SOFT7044) and the Linux Administration (COMP7036) modules.

PHP was studied and practiced in the Server-side Web Frameworks module (COMP8001) and the author expanded on this while interning at Nimbus during the Year Three Work Placement Program working on various projects under their supervisionf.

There are several new techniques that will be used in this project that go beyond the foundations laid down in CIT. The author plans on using the Web Scraping Python tool BeautifulSoup to scrape and parse news data off the popular technology news sites that come recommended by Nimbus to keep abreast of news articles relating to companies in their Contacts database. For the conversion of excel spreadsheet files, the author’s research shows that using the xlsxworker library to parse and extract the Budget data from the CIT excel spreadsheets into the MySQL Database is a viable strategy.

Data Visualisation will be studied in the second semester of Year Four within the Interactive Data Visualisation (COMP8054) module and it is hoped that this module will provide a grounding for the Dashboard module within the project.

Agile and in particular Scrum will be used to manage this project, particularly within the implementation stage. This strategy emanates from the Agile Processes (COMP7039) module. Also, from this module,  GitHub will be used for source control/version management and Jenkins for CI/CD.

\section{Structure of This Document}
% notice how I cross referenced the chapters through using the \label tag --> LaTeX is VERY similar to HTML and other mark up languages so you should see nothing new here!
This section is quite formulaic. Briefly describe the structure of this document, enumerating what does each chapter and section stands for. For instance in this work in Chapter \ref{chap:background} the guidance in structuring the literature review is given. Chapter \ref{chap:problem} describes the main requirements for the problem definition and so on ...